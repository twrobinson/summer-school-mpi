\documentclass[aspectratio=43]{beamer}
\usepackage{ragged2e}
\usepackage{multirow}
\usepackage{alltt}

\usetheme{CSCS}


\newcommand{\SummerSchoolYear}{2017}
\newcommand{\SummerSchoolDate}{July 19--20}
\newcommand{\SummerSchoolAuthor}{Tim Robinson}

\newcommand{\footlinetext}{Summer School \SummerSchoolYear{} -- MPI}

\author{\SummerSchoolAuthor, CSCS}
\title{Message Passing Interface (MPI)}
\subtitle{Summer School \SummerSchoolYear{}  -- Effective High Performance Computing}
\date{\SummerSchoolDate, \SummerSchoolYear}


% Select the image for the title page
%\newcommand{\picturetitle}{cscs_images/image3.pdf}
\newcommand{\picturetitle}{cscs_images/image5.pdf}
%\newcommand{\picturetitle}{cscs_images/image6.pdf}

\begin{document}

% TITLE SLIDE
\cscstitle

\begin{frame}{Course Objectives}
\begin{itemize}
\item Hybrid OpenMP/MPI, preparing MPI for OpenMP
\end{itemize}
\end{frame}

% TABLE OF CONTENT SLIDE
\cscstableofcontents[hideallsubsections]{General Course Structure}

\section{An introduction to MPI}
\section{Point-to-point communications}
\section{Collective communications}
\section{Topology}
\section{Datatypes}
\section{Other topics}
\cscstableofcontents[currentsection]{General Course Structure}

% CHAPTER SLIDE
\cscschapter{Configure MPI to enable OpenMP}

\subsection{Hybrid OpenMP/MPI}

\begin{frame}[fragile]{Why OpenMP with MPI}
There are two main motivations:
\begin{itemize}
\item To reduce the memory footprint (both in application and in communication buffers)
\item To increase scalability (but you might not be faster than with MPI alone)
\end{itemize}
\end{frame}

\begin{frame}[fragile]{Reducing memory requirements}
\begin{itemize}
\item Memory per core is generally decreasing
\item MPI applications require some data to be replicated between MPI processes 
 \begin{itemize}
        \item Read-only lookup table where every process has a copy
        \item Halo regions of neighbours          
        \item MPI internal data structions, e.g. communicaiton buffers
    \end{itemize}
\end{itemize}
\begin{center}
\begin{tabular}{|c|c|c|}
\hline
Local domain size &  halos & \% data in halos\\\hline
$100^3 = 1,000,000$  & $102^3 - 100^3 = 61,208$ & $6\%$\\\hline
$50^3 = 125,000$  &  $52^3 - 50^3 = 15,608$ & $11\%$ \\\hline
$20^3 = 8,000$  & $22^3 - 20^3 = 2648$  & $25\%$ \\\hline
\end{tabular}
\end{center}
\end{frame}

\begin{frame}[fragile]{Improving performance}
\begin{itemize}
\item In the regime where MPI is scaling well, OpenMP will introduce overhead
\item OpenMP can be used to exploit lower levels of parallelism
 \begin{itemize}
        \item Might be hard to load balance with MPI
        \item Might have irregular communication pattern
    \end{itemize}
\item In some cases collective communication overheads might be reduced
\item OpenMP has in-built load balancing capabilities (loop schedules, tasks)
\end{itemize}
\end{frame}


\begin{frame}[fragile]{Preparing MPI for OpenMP}
MPI requires to be setup with threads enabled:
\begin{itemize}
\item \lstinlinePseudo{MPI_Init} should be replaced by \lstinlinePseudo{MPI_Init_thread}
\end{itemize}
\begin{Pseudolisting}[]{}
MPI_Init_thread(required, provided, ierror)
\end{Pseudolisting}

\lstinlinePseudo{required} specifies the requested level of thread support, and the actual level of support is then returned into \lstinlinePseudo{provided}.\\
You should check the value of \lstinlinePseudo{provided} after the call
\end{frame}

\begin{frame}[fragile]{MPI Thread level}
\begin{itemize}
    \item \lstinlinePseudo{MPI_THREAD_SINGLE}: Only one thread will execute (MPI-only application)
    \item \lstinlinePseudo{MPI_THREAD_FUNNELED}: Only master thread will make MPI calls (master = the thread calling \lstinlinePseudo{MPI_Init_thread})
    \item \lstinlinePseudo{MPI_THREAD_SERIALIZED}: Only one thread at a time will make MPI calls (user responsibility)
    \item \lstinlinePseudo{MPI_THREAD_MULTIPLE}: Any thread may call MPI at any time, however that leads to slower performance (lock mechanism in MPI)
\end{itemize}
\end{frame}

\begin{frame}[fragile]{Checking the level of thread support provided}
\begin{itemize}
\item The four classes of support are guaranteed to be monotonicly increasing
 \lstinlinePseudo{MPI_THREAD_SINGLE} $<$ \lstinlinePseudo{MPI_THREAD_FUNNELED} $<$ \lstinlinePseudo{MPI_THREAD_SERIALIZED} $<$ \lstinlinePseudo{MPI_THREAD_MULTIPLE}
\end{itemize}
\begin{itemize}
\item Therefore you can test to make sure at least the level you require is provided and exit if not:
\end{itemize}
\begin{Pseudolisting}[]{}
if (provided < requested) {
     printf("Not high enough level of thread support.\n");
     MPI_Abort(MPI_COMM_WORLD,1);
}
\end{Pseudolisting}
\end{frame}


% THANK YOU SLIDE
\cscsthankyou{Thank you for your attention.}

\end{document}
