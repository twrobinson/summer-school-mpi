\documentclass[aspectratio=43]{beamer}
\usepackage{ragged2e}
\usepackage{multirow}

\usetheme{CSCS}


\newcommand{\SummerSchoolYear}{2017}
\newcommand{\SummerSchoolDate}{July 19--20}
\newcommand{\SummerSchoolAuthor}{Tim Robinson}

\newcommand{\footlinetext}{Summer School \SummerSchoolYear{} -- MPI}

\author{\SummerSchoolAuthor, CSCS}
\title{Message Passing Interface (MPI)}
\subtitle{Summer School \SummerSchoolYear{}  -- Effective High Performance Computing}
\date{\SummerSchoolDate, \SummerSchoolYear}



% Select the image for the title page
%\newcommand{\picturetitle}{cscs_images/image3.pdf}
\newcommand{\picturetitle}{cscs_images/image5.pdf}
%\newcommand{\picturetitle}{cscs_images/image6.pdf}

\begin{document}

% TITLE SLIDE
\cscstitle

\begin{frame}{Previous course summary}
\begin{itemize}
\item Point-to-point communication
\item Blocking and non-blocking communication
\item Transfer modes
\end{itemize}
\end{frame}

\begin{frame}{Course Objectives}
\begin{itemize}
\item The understanding of a collective operations
\item Knowledge of the different collective operations
\end{itemize}
\end{frame}

% TABLE OF CONTENT SLIDE
\cscstableofcontents[hideallsubsections]{General Course Structure}

\section{An introduction to MPI}
\section{Point-to-point communications}
\section{Collective communications}
\cscstableofcontents[currentsection]{General Course Structure}

% CHAPTER SLIDE
\cscschapter{Collective communications}

\subsection{Collective communications}

\begin{frame}{Collective operations}
Communications involving a group of processes part of a communicator.
Different algorithms: $1\rightarrow N$, $N\rightarrow 1$ or $N\rightarrow N\;$ ($1\rightarrow 1$ = pt2pt).\\

Example:
\begin{itemize}
\item Barrier Synchronization
\item Broadcast
\item Gather/Scatter
\item AlltoAll
\item Reduction (sum, max, prod, \ldots )
\end{itemize}

Features:
\begin{itemize}
    \item All processes must call the collective routine, one is the root
    \item No tags
\end{itemize}

The MPI library should use the most efficient communication algorithm for the particular platform.
\end{frame}

\begin{frame}{Collective operations schemes}
\begin{center}
    \includegraphics[scale=0.32]{03.MPI_Coll/collop.pdf}
\end{center}
\end{frame}

\subsection{Barrier}

\begin{frame}[fragile]{Barrier}
Stop processes until all processes within a communicator reach the barrier.\\
\begin{Pseudolisting}[]{}
MPI_Barrier(comm)
\end{Pseudolisting}
\begin{center}
\includegraphics[scale=0.5]{03.MPI_Coll/barrier.pdf}
\end{center}
\end{frame}

\subsection{Broadcast}
\begin{frame}[fragile]{Broadcast}
One-to-all communication: same data sent from root process to all other processes in the communicator.\\
\begin{Pseudolisting}[]{}
MPI_Bcast(buf, count, type, root, comm)
\end{Pseudolisting}
\begin{black1block}{}
\begin{tabular}{rp{8cm}}
\textbf{root} & rank being the initiator of the collective operation\\
\end{tabular}
\end{black1block}
\end{frame}

\subsection{Scatter/Gather}

\begin{frame}[fragile]{Scatter}
One-to-all communication: different data sent from the root process to all other processes in the communicator.\\
\begin{Pseudolisting}[]{}
MPI_Scatter(sndbuf, sndcount, sndtype,
            rcvbuf, rcvcount, rcvtype, root, comm)
\end{Pseudolisting}
\begin{black1block}{}
\begin{tabular}{rp{8cm}}
    \textbf{sndcount} & number of elements sent to each process, {\textbf{not the size of sndbuf}}, that should be sndcount times the number of process in the communicator\\
\textbf{rcvcount} & number of element in the receive buffer\\
\end{tabular}
\end{black1block}
The sender arguments are meaningful only for root.
\end{frame}

%\begin{frame}[fragile]{Scatter with different buffers size}
%One-to-all communication: Scatter + distributes individual messages from root to each process in communicator. Messages can have different sizes and displacements.\\
%\begin{Pseudolisting}[]{}
%CALL MPI_SCATTERV(sndbuf, sndcount, displs, sndtype,
%                  rcvbuf, rcvcount, rcvtype,
%                  root, comm, ierr)
%\end{Pseudolisting}
%\begin{black1block}{}
%\begin{tabular}{rp{8cm}}
%\textbf{displs} & (INTEGER) entry i specifies the displacement (relative to sendbuf) from which to take the outgoing data to process i.\\
%\end{tabular}
%\end{black1block}
%\begin{center}
%\includegraphics[scale=0.33]{03.MPI_Coll/scatterv.pdf}
%\end{center}
%\end{frame}

%\begin{frame}{Scatter with different buffers size}
%\includegraphics[scale=0.5]{03.MPI_Coll/scatter2.pdf}
%\end{frame}

\begin{frame}[fragile]{Gather}
One-to-all communication: different data collected by the root process, from all others processes in the communicator.\\
\begin{Pseudolisting}[]{}
MPI_Gather(sndbuf, sndcount, sndtype,
           rcvbuf, rcvcount, rcvtype, root, comm)
\end{Pseudolisting}
\begin{black1block}{}
\begin{tabular}{rp{8cm}}
    \textbf{rcvcount} & the number of elements collected from each process, {\textbf{not the size of rcvbuf}}, that should be rcvcount times the number of process in the communicator\\
\textbf{sndcount} & number of element in the send buffer\\
\end{tabular}
\end{black1block}
The receive arguments are meaningful only for root.
\end{frame}

%\begin{frame}{Gather with identical buffer size}
%\includegraphics[scale=0.5]{03.MPI_Coll/gather1.pdf}
%\end{frame}

%\begin{frame}[fragile]{Gather with different buffers size}
%One-to-all communication: Gather + collects individual messages from each process in communicator to the root process and store them in rank order. Messages can have different sizes and displacements.\\
%\begin{Pseudolisting}[]{}
%CALL MPI_GATHERV(sndbuf, sndcount, sndtype,
%                 rcvbuf, rcvcount, displs, rcvtype,
%                 root, comm, ierr)
%\end{Pseudolisting}
%\begin{black1block}{}
%\begin{tabular}{rp{8cm}}
%\textbf{displs} & (INTEGER) entry i specifies the displacement (relative to sendbuf) from which to take the outgoing data to process i.\\
%\end{tabular}
%\end{black1block}
%\begin{center}
%\includegraphics[scale=0.33]{03.MPI_Coll/gatherv.pdf}
%\end{center}
%\end{frame}
%
%\begin{frame}{Gather with different buffers size}
%\includegraphics[scale=0.5]{03.MPI_Coll/gather2.pdf}
%\end{frame}

\subsection{All to all}
\begin{frame}[fragile]{Global exchange: All to All}
All-to-all communication: global exchange, all processes exchange their data. Useful for data transposition.\\
\begin{Pseudolisting}[]{}
MPI_Alltoall(sndbuf, sndcount, sndtype,
             rcvbuf, rcvcount, rcvtype, comm)
\end{Pseudolisting}
\end{frame}


\subsection{Reduction}
\begin{frame}{Reduction}
The reduction operation allows to:
\begin{itemize}
\item Collect data from each process
\item Reduce the data to a single value
\item Store the result on the root processes
\item Store the result on all processes
\item Overlap communication and computation
\end{itemize}
\end{frame}

\begin{frame}[fragile]{Reduction}
\begin{Pseudolisting}[]{}
MPI_Reduce(sndbuf, rcvbuf, count, type, op, root, comm)
\end{Pseudolisting}
\begin{black1block}{}
\begin{tabular}{rp{8cm}}
\textbf{op} & parallel operation to perform\\
\end{tabular}
\end{black1block}
\begin{center}
\includegraphics[scale=0.5]{03.MPI_Coll/reduce.pdf}
\end{center}
\end{frame}

\begin{frame}[fragile]{Reduction operators}
\begin{center}
\begin{tabular}{|c||c|}
    \hline
    \color{cscsblue}\textbf{MPI op} & \color{cscsbrown}\textbf{Operation} \\\hline\hline
    \verb+MPI_MAX+ & Maximum\\\hline
    \verb+MPI_MIN+ & Minimum\\\hline
    \verb+MPI_SUM+ & Sum\\\hline
    \verb+MPI_PROD+ & Product\\\hline
    \verb+MPI_LAND+ & Logical AND\\\hline
    \verb+MPI_BAND+ & Bitwise AND\\\hline
    \verb+MPI_LOR+ & Logical OR\\\hline
    \verb+MPI_BOR+ & Bitwise OR\\\hline
    \verb+MPI_LXOR+ & Logical exclusive OR\\\hline
    \verb+MPI_BXOR+ & Bitwise exclusive OR\\\hline
    \verb+MPI_MAXLOC+ & Maximum and location\\\hline
    \verb+MPI_MINLOC+ & Minimum and location\\\hline
\end{tabular}
\end{center}
\end{frame}

\subsection{Global collective operations}
\begin{frame}[fragile]{Global collective operations}
The result of the one-to-all operation is known by all ranks at the end of the operation.
\begin{Pseudolisting}[]{}
MPI_Allgather(sndbuf, sndcount, sndtype,
              rcvbuf, rcvcount, rcvtype, comm)

MPI_Allreduce(sndbuf, rcvbuf, count, type, op, comm)
\end{Pseudolisting}
The argument \textbf{root} is missing, the result is stored in all processes.\\
\end{frame}

\subsection{Non-blocking coll-op}
\begin{frame}[fragile]{Non-blocking collective operations}
All collective operations have a non-blocking version.\\
Example:
\begin{Pseudolisting}[]{}
MPI_Ibcast(buf, count, type, root, comm, request)
\end{Pseudolisting}
Other functions:\\
\begin{Pseudolisting}[]{}
MPI_Ibarrier, MPI_Igather, MPI_Ireduce, MPI_Iscatter,
MPI_Iallgather, MPI_Iallreduce, MPI_Ialltoall
\end{Pseudolisting}
\end{frame}

\begin{frame}[fragile]{Other functions}
\begin{itemize}
    \item Operations with different buffer sizes:\\\hspace{1cm}\lstinlinePseudo{MPI_AlltoAllv, MPI_Gatherv, MPI_Scatterv, MPI_Allgatherv}
    \item Neighbor operations, based on topology:\\\hspace{1cm}\lstinlinePseudo{MPI_Neighbor_gather, MPI_Neighbor_alltoall}
    \item Cummulative per rank reduction:\\\hspace{1cm}\lstinlinePseudo{MPI_Scan, MPI_Exscan}
    \item Create your own operator:\\\hspace{1cm}\lstinlinePseudo{MPI_Op_create, MPI_Op_free}
\end{itemize}
\end{frame}

\begin{frame}{Practicals}
    \begin{brown2block}{Exercise: 03.MPI\_Coll}
    \begin{enumerate}
        \item Read from the terminal and broadcast the input
        \item Initialise an array and scatter it
        \item Reduction operation 
        \item Reduction with results stored in all ranks (allreduce)
    \end{enumerate}
    \end{brown2block}
\end{frame}

\section{Topology}
\section{Datatypes}

% THANK YOU SLIDE
\cscsthankyou{Thank you for your attention.}

\end{document}
